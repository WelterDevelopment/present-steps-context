\startproduct[prd_presentation]
\environment[env_presentation]
% If you want to use your own layout, make sure you set the following lines since the stepping relies on *not* having the section head in the page content. Instead, the Section/Subsection gets printed in the page header
  %\setuphead[section,subsection][style={\bfc}, before={}, after={}, page=yes, placehead=empty]
  %\setupheadertexts[{\getmarking[sectionnumber] \getmarking[section] -- \getmarking[subsectionnumber] \getmarking[subsection]}][]


\usemodule[present-steps] % modified

\useURL[thomas][mailto:wdev@outlook.de][][wdev@outlook.de]

\setvariables[meta][title={Presentations in \ConTeXt}, subtitle={Stepping with adjustments to the original s-present-steps.mkiv}, author={Thomas Welter}, date={2020.05.26}]

\setupinteraction[title=\getvariable{meta}{title}, subtitle=\getvariable{meta}{subtitle}, author=\getvariable{meta}{author}, date=\getvariable{meta}{date}]

\setupsubpagenumber[way=bysubsection, state=start, numberconversion=characters] % subpage counting is just for fun in this example, but might be usable for some komplex interactionbar visualization

\starttext

\startfrontmatter
\definemakeup[title][doublesided=no, page=right, headerstate=empty, footerstate=empty,  pagestate=start, style=bigbodyfont, align=middle]
\starttitlemakeup
    {\bfd \getvariable{meta}{title}}
    \blank[3*medium]
    {\tfb \getvariable{meta}{subtitle}}
    \blank[3*medium]
    {\tfa \getvariable{meta}{author}}
    \blank[2*medium]
    {\tfa \getvariable{meta}{date}}
    \blank[3*medium]
    {\tfa \from[thomas]}
\stoptitlemakeup

\setupbackgrounds[header][text][frame=off, bottomframe=on]

\setmarking[section]{Content} % \setmarking[section]{} will set the header text without calling \section{}
\bookmark[section]{Content}
\placecontent
\stopfrontmatter

\startbodymatter
\startsection[title=Stepping is cool]
\startsubsection[title=Basic stepping]
\startsteps
	\startstep
        STEP ONE. 

        Step.Substep :  \currentstep.\currentsubstep 

        Page.SubPage :  \userpagenumber.\subpagenumber

        \blank
	\stopstep

	\startstep[n=7]
		STEP Two. Note that the automatic increment of \type{\pagenumber} is stopped during stepping (see \type{\setuppagenumber[state=stop]} in the module). Instead of incrementing \type{\pagenumber}, a \type{\subpagenumber} (here shown in characters) is used, which gets resetted by subsection.

		Step.Substep :  \currentstep.\currentsubstep 
		\startitemize[horizontal,two,packed,broad]
            \startsubstep \startitem 1 \stopitem \stopsubstep
            \startsubstep \startitem 2 \stopitem \stopsubstep
            \startsubstep \startitem 3 \stopitem \stopsubstep
            \startsubstep \startitem 4 \stopitem \stopsubstep
            \startsubstep \startitem 5 \stopitem \stopsubstep
            \startsubstep \startitem 6 \stopitem \stopsubstep
            \startsubstep \startitem 7 \stopitem \stopsubstep
		\stopitemize		
	\stopstep

	\startstep
		STEP THREE
	\stopstep

	\startstep
		STEP FOUR
	\stopstep
\stopsteps
\stopsubsection

\startsubsection[title=Columns]
\startsteps
    \startstep
        Some text that is not in columns
    \stopstep
    \startstep[n=2]
        \startcolumnset[n=2]
            \startsubstep 
                first col
            \stopsubstep
            \startsubstep 
                \column second col 
            \stopsubstep
        \stopcolumnset
    \stopstep
\stopsteps
\stopsubsection

\startsubsection[title=Tables]
\setupTABLE[r][each][style=\tfx\it, align=center]
\startsteps
    \startstep
        Tables are also working
    \stopstep
    \startstep[n=3]
        \startplacetable[location=here, reference=tab:exampletable, title={Example table}]
            \bTABLE[split=repeat,option=stretch]
            \startsubstep
                \bTABLEhead
                \bTR
                \bTH  head1 \eTH
                \bTH  head2 \eTH
                \eTR
                \eTABLEhead
            \stopsubstep
            \bTABLEbody
            \startsubstep
                \bTR \bTD One \eTD \bTD two \eTD \eTR
            \stopsubstep
            \startsubstep
                \bTR \bTD One \eTD \bTD two \eTD \eTR
            \stopsubstep
            \eTABLEbody
            \eTABLE
        \stopplacetable
        \setupcaptions[table][state=stop] % Quick hack do disable the table counter
    \stopstep
    \startstep 
        Some more text
    \stopstep
\stopsteps
\stopsubsection
\stopsection
\stopbodymatter

\startappendices
\section{Some appendices}
...
\stopappendices

\startbackmatter
\section{Some backup slides}
...
\stopbackmatter

\stoptext

\stopproduct
